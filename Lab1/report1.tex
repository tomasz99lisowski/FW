\documentclass{article} % lub inna klasa dokumentu, np. report lub book
\usepackage[utf8]{inputenc} % ustawienie kodowania na UTF-8
\usepackage{amsmath, amssymb, amsthm} % biblioteki do działań matematycznych
\usepackage[T1]{fontenc}    % wybór odpowiedniego kodowania czcionek
\usepackage{polski}         % dodanie wsparcia dla polskich znaków
\usepackage{babel}          % automatyczne dostosowanie języka

\title{Sprawozdanie}

\author{Tomasz Lisowski 197749\and Filip Świniarski 197725\and Nikodem Miłuch 197922}

\begin{document}

\maketitle

\section{Wstęp}

Celem ćwiczenia było wyznaczenie gęstości dwóch wybrancyh cieczy: denaturatu i słonej wody przy użyciu wagi Mohra oraz piknometru. Ciecz wzorcową w układzie badawczym stanowiła woda destylowana.

\subsection{Waga Mohra}

Waga Mohra to precyzyjne urządzenie służące do wyznaczania gęstości cieczy i małych ciał stałych. Składa się z poziomej belki z podziałką oraz obciążników, które umożliwiają precyzyjne wyważenie. Na belce umieszczone jest szkiełko zanurzeniowe (aerometr), które jest zanurzane w badanej cieczy. Gęstość cieczy wyznacza się na podstawie położenia obciążników, które równoważą siłę wyporu cieczy działającą na zanurzone szkiełko.

\subsection{Piknometr}

Piknometr to precyzyjne naczynie laboratoryjne, w kształcie małej butelki z wąską szyjką, służące do pomiaru gęstości cieczy oraz ciał stałych w formie proszku. Piknometr ma dokładnie określoną objętość, co umożliwia wyznaczenie gęstości poprzez zważenie piknometru wypełnionego badaną substancją, a następnie zważenie go z wodą (dla odniesienia). Dzięki tej metodzie można dokładnie obliczyć gęstość badanej substancji na podstawie różnicy mas.


\section{Gęstość cieczy mierzona za pomocą Wagi Mohra}

\subsection{Opis doświadczenia}

Zbieranie pomiarów przy użyciu wagi Mohra polegało na wyznaczaniu gęstości cieczy poprzez zrównoważenie siły wyporu. Proces wyglądał następująco:

\begin{enumerate}
    \item Na początku wagę należy ustawić w pozycji równowagi.
    \item Do badanego roztworu zanurzano specjalny odważnik zawieszony na końcu belki.
    \item Na belce wagi umieszczano obciążniki w odpowiednich punktach podziałki, aby zrównoważyć siłę wyporu działającą na odważnik.
    \item Na podstawie położenia obciążników (koników) na podziałce belki wyznaczano wartość odpowiadającą gęstości cieczy.
\end{enumerate}


Cała procedura pozwalała na bezpośrednie odczytanie gęstości badanej cieczy, dzięki zrównoważeniu ciężarów.

\subsection{Wyprowadzenie wzorów}

Po zanurzeniu ciężarka w cieczy wzorcowej o znanej gęstości $p_w$ (np. w wodzie
destylowanej) zacznie na niego działać siła wyporu równa, zgodnie z prawem
Archimedesa, ciężarowi cieczy wypartej przez ciężarek:
{\large
    \begin{equation}
    F_w = p_wgV    
    \end{equation}
}
gdzie $V$ - objętość ciężarka, $g$ - przyspieszenie grawitacyjne.  Dla przywrócenia równowagi (po zanurzeniu nurka w cieczy „wzorcowej”) zawieszamy na odpowiednich haczykach wagi o numerach $p_w$, 
$q_w$ i $r_w$ odpowiednie koniki o masach m, m/10, m/100. Po przywróceniu równowagi,
moment siły wyporu działającej na ciężarek zawieszony na dziesiątym haczyku,
a więc w odległości $L$ od punktu podparcia dźwigni, zostanie zrównoważony sumą momentów sił ciężkości działających na poszczególne koniki, zawieszone w
odległościach $p_w \frac{L}{10}$ , $q_w \frac{L}{10}$ , $r_w \frac{L}{10}$ od punktu podparcia dźwigni:
{\large
    \begin{equation}
    p_wVg = mg(\frac{p_w}{10}+\frac{q_w}{10}\frac{1}{10}+\frac{r_w}{10}\frac{1}{100})
    \end{equation}
}
analogicznie dla pomiarów z badaną cieczą otrzymujemy:
{\large
    \begin{equation}
    p_wVg = mg(\frac{p}{10}+\frac{q}{10}\frac{1}{10}+\frac{r}{10}\frac{1}{100})
    \end{equation}
}
dzieląc równianie (2) przez równianie (3) otrzymujemy wzór na gęstość badanej cieczy:
{\large
    \begin{equation}
    p = p_w\frac{p+\frac{q}{10}+\frac{r}{100}}{p_w+\frac{q_w}{10}+\frac{r_w}{100}}
    \end{equation}
}
\subsection{Pomiary}

Pomiary doświadczenia zebrano w tabeli poniżej:

\begin{table}[h!]
\centering
\begin{tabular}{|c|c|c|c|}
\hline
\textbf{Ciecz} & \textbf{M1} & \textbf{M10} & \textbf{M100} \\
\hline
Woda destylowana 1 & 1,9 & 2 & 0 \\
Woda destylowana 2 & 1,9 & 0 & 4 \\
Woda solona & 2,9 & 2 & 5 \\
Denaturat & 8 & 1 & 5,4 \\
\hline
\end{tabular}
\caption{Położenie odważników na wadze}
\label{table:students}
\end{table}

\subsection{Niepewności pomiarowe}

Niepewności pomiarową $u(f)$ wyznaczymy jako niepwność złożoną z położenia każdego ciężarka. Przyjmujemy, że $\Delta r = \Delta r_w = 0.5$. Pochodne cząstkowe poszczególnych położeń przyjmą wartość:
{\
    \begin{equation}
        \frac{\partial f}{\partial p} = \frac{p_w}{A}
        \quad\frac{\partial f}{\partial q} = \frac{p_w}{10A}
        \quad\frac{\partial f}{\partial r} = \frac{p_w}{100A}
    \end{equation}
}
gdzie:
{\
    \begin{equation}
        A = p_w + \frac{q_w}{10} + \frac{r_w}{100}
    \end{equation}
}

Dla pomiaru wzorcowego niepewności obliczymy w następujący sposób:

{\large
    \begin{equation}
        u(f) = \sqrt{(\frac{p_w}{A}\Delta r_w)^2+(\frac{p_w}{10A}\Delta r_w)^2+(\frac{p_w}{100A}\Delta r_w)^2}
    \end{equation}
}
gdzie wartość $A$ wyznaczymy z równania (6), a $p_w, q_w, r_w$ to wartości dla cieczy wzorcowej.
\subsection{Wyniki}
Wyniki pomiarów zebrano w tabeli poniżej:

\begin{table}[h!]
\centering
\begin{tabular}{|c|c|c|c|}
\hline
\textbf{Ciecz} & \textbf{M1} & \textbf{M10} & \textbf{M100} \\
\hline
Woda destylowana 1 & 1,9 & 2 & 0 \\
Woda destylowana 2 & 1,9 & 0 & 4 \\
Woda solona & 2,9 & 2 & 5 \\
Denaturat & 8 & 1 & 5,4 \\
\hline
\end{tabular}
\caption{Wyniki pomiarów}
\label{table:students}
\end{table}


\end{document}
