\documentclass{article} % lub inna klasa dokumentu, np. report lub book
\usepackage[utf8]{inputenc} % ustawienie kodowania na UTF-8
\usepackage{amsmath, amssymb, amsthm} % biblioteki do działań matematycznych
\usepackage[T1]{fontenc}    % wybór odpowiedniego kodowania czcionek
\usepackage{polski}         % dodanie wsparcia dla polskich znaków
\usepackage{babel}          % automatyczne dostosowanie języka

\title{Sprawozdanie}

\author{Tomasz Lisowski 197749\and Filip Świniarski 197725\and Nikodem Miłuch 197922}

\begin{document}

\maketitle

\section{Wstęp}

Celem ćwiczenia było wyznaczenie gęstości dwóch wybrancyh cieczy: denaturatu i słonej wody przy użyciu wagi Mohra oraz piknometru. Ciecz wzorcową w układzie badawczym stanowiła woda destylowana.

\subsection{Waga Mohra}

Waga Mohra to precyzyjne urządzenie służące do wyznaczania gęstości cieczy i małych ciał stałych. Składa się z poziomej belki z podziałką oraz obciążników, które umożliwiają precyzyjne wyważenie. Na belce umieszczone jest szkiełko zanurzeniowe (aerometr), które jest zanurzane w badanej cieczy. Gęstość cieczy wyznacza się na podstawie położenia obciążników, które równoważą siłę wyporu cieczy działającą na zanurzone szkiełko.

\subsection{Piknometr}

Piknometr to precyzyjne naczynie laboratoryjne, w kształcie małej butelki z wąską szyjką, służące do pomiaru gęstości cieczy oraz ciał stałych w formie proszku. Piknometr ma dokładnie określoną objętość, co umożliwia wyznaczenie gęstości poprzez zważenie piknometru wypełnionego badaną substancją, a następnie zważenie go z wodą (dla odniesienia). Dzięki tej metodzie można dokładnie obliczyć gęstość badanej substancji na podstawie różnicy mas.


\section{Gęstość cieczy mierzona za pomocą Wagi Mohra}

\subsection{Opis doświadczenia}

Zbieranie pomiarów przy użyciu wagi Mohra polegało na wyznaczaniu gęstości cieczy poprzez zrównoważenie siły wyporu. Proces wyglądał następująco:

\begin{enumerate}
    \item Na początku wagę należy ustawić w pozycji równowagi.
    \item Do badanego roztworu zanurzano specjalny odważnik zawieszony na końcu belki.
    \item Na belce wagi umieszczano obciążniki w odpowiednich punktach podziałki, aby zrównoważyć siłę wyporu działającą na odważnik.
    \item Na podstawie położenia obciążników (koników) na podziałce belki wyznaczano wartość odpowiadającą gęstości cieczy.
\end{enumerate}


Cała procedura pozwalała na bezpośrednie odczytanie gęstości badanej cieczy, dzięki zrównoważeniu ciężarów.

\subsection{Wyprowadzenie wzorów}

Po zanurzeniu ciężarka w cieczy wzorcowej o znanej gęstości $p_w$ (np. w wodzie
destylowanej) zacznie na niego działać siła wyporu równa, zgodnie z prawem
Archimedesa, ciężarowi cieczy wypartej przez ciężarek:
{\large
    \begin{equation}
    F_w = p_wgV    
    \end{equation}
}
gdzie $V$ - objętość ciężarka, $g$ - przyspieszenie grawitacyjne.  Dla przywrócenia równowagi (po zanurzeniu nurka w cieczy „wzorcowej”) zawieszamy na odpowiednich haczykach wagi o numerach $p_w$, 
$q_w$ i $r_w$ odpowiednie koniki o masach m, m/10, m/100. Po przywróceniu równowagi,
moment siły wyporu działającej na ciężarek zawieszony na dziesiątym haczyku,
a więc w odległości $L$ od punktu podparcia dźwigni, zostanie zrównoważony sumą momentów sił ciężkości działających na poszczególne koniki, zawieszone w
odległościach $p_w \frac{L}{10}$ , $q_w \frac{L}{10}$ , $r_w \frac{L}{10}$ od punktu podparcia dźwigni:
{\large
    \begin{equation}
    \rho_wVg = mg(\frac{p_w}{10}+\frac{q_w}{10}\frac{1}{10}+\frac{r_w}{10}\frac{1}{100})
    \end{equation}
}
analogicznie dla pomiarów z badaną cieczą otrzymujemy:
{\large
    \begin{equation}
    \rho_wVg = mg(\frac{p}{10}+\frac{q}{10}\frac{1}{10}+\frac{r}{10}\frac{1}{100})
    \end{equation}
}
dzieląc równianie (2) przez równianie (3) otrzymujemy wzór na gęstość badanej cieczy:
{\large
    \begin{equation}
    \rho = \rho_w\frac{p+\frac{q}{10}+\frac{r}{100}}{p_w+\frac{q_w}{10}+\frac{r_w}{100}}
    \end{equation}
}

\subsection{Niepewność pomiarowa}

Niepewność pomiarową $u(f)$ wyznaczymy jako niepwność złożoną z położenia każdego ciężarka. Przyjmujemy, że $\Delta r = \Delta r_w = 0.5$. Pochodne cząstkowe poszczególnych położeń przyjmą wartość:
{\
    \begin{equation}
        \frac{\partial f}{\partial p} = \frac{p_w}{A}
        \quad\frac{\partial f}{\partial q} = \frac{p_w}{10A}
        \quad\frac{\partial f}{\partial r} = \frac{p_w}{100A}
    \end{equation}
}
gdzie:
{\
    \begin{equation}
        A = p_w + \frac{q_w}{10} + \frac{r_w}{100}
    \end{equation}
}

Dla pomiaru wzorcowego niepewności obliczymy w następujący sposób:

{\large
    \begin{equation}
        u(f) = \sqrt{(\frac{\rho_w}{A}\Delta r_w)^2+(\frac{\rho_w}{10A}\Delta r_w)^2+(\frac{\rho_w}{100A}\Delta r_w)^2}
    \end{equation}
}
gdzie wartość $A$ wyznaczymy z równania (6), a $p_w, q_w, r_w$ to wartości dla cieczy wzorcowej.
\subsection{Wyniki pomiarów}
Wyniki pomiarów zebrano w tabeli poniżej. Jeżeli w danej komórce pojawiają się dwie wartości, oznacza to, że do zrównoważenia wagi użyto dwóch ciężarków tego rodzaju. Nie wpływa to na sposób wyznaczania wzorów (4) i (7) a jeddynie powoduje uwzględnieine dotakowego wyrazu o odpowiednim współczynniku.

\begin{table}[h!]
\centering
\begin{tabular}{|c|c|c|c|}
\hline
\textbf{Ciecz} & \textbf{M1} & \textbf{M10} & \textbf{M100} \\
\hline
Woda destylowana 1 & 1,9 & 2 & 0 \\
Woda destylowana 2 & 1,9 & 0 & 4 \\
Woda solona & 2,9 & 2 & 5 \\
Denaturat & 8 & 1 & 5,4 \\
\hline
\end{tabular}
\caption{Wyniki pomiarów}
\label{table:students}
\end{table}


\subsection{Wyniki doświadczenia}

Ostateczne wyniki pomiarów gęstości cieczy wraz z niepwnościami pomiarowymi zależą od przyjęcia pewnych danych wzorcowych. Dla poniższych pomiarów przyjęto gęstość wody $\rho_w = 1000kg/m^3$. Wartości wyznaczono poprzez wzory (4) i (7).
\subsubsection{Woda destylowana 1 jako ciecz wzorcowa}
Dla Wody destylowanej 1 jako cieczy wzorcowej wyniki pomiarów gęstości przyjmują następujące wartości:
\begin{table}[h!]
\centering
\begin{tabular}{|c|c|c|c|}
\hline
\textbf{Ciecz} & \textbf{Gęstość[$kq/m^3$]} & \textbf{Błąd pomiarowy[$kq/m^3$]}\\
\hline
Woda solona & 1102,9 & 49,26\\
Denaturat & 802,9 & 49,26\\
\hline
\end{tabular}
\caption{Wyniki pomiarów}
\label{table:students}
\end{table}

\subsubsection{Woda destylowana 2 jako ciecz wzorcowa}
Dla Wody destylowanej 2 jako cieczy wzorcowej wyniki pomiarów gęstości przyjmują następujące wartości:
\begin{table}[h!]
\centering
\begin{tabular}{|c|c|c|c|}
\hline
\textbf{Ciecz} & \textbf{Gęstość[$kq/m^3$]} & \textbf{Błąd pomiarowy[$kq/m^3$]}\\
\hline
Woda solona & 1120,5 & 50,051\\
Denaturat & 825,7 & 50,051\\
\hline
\end{tabular}
\caption{Wyniki pomiarów}
\label{table:students}
\end{table}

\subsection{Wnioski}

Pomiary przeprowadzone wagą Mohra wskazały poprawne wartości gęstości denaturatu i solonej wody. Duży błąd pomiarowy (w przypadku denaturatu wynoszący więcej niż 5\%) wynika z wad mechanicznych wagi udostępnionej w czasie laboratorium.

\section{Gęstość cieczy mierzona za pomocą piknometru}
\subsection{Opis doświadczenia}
Wyznaczanie gęstości cieczy odbywa się w następujący sposób:
\begin{enumerate}
    \item Wyznaczamy masę piknometru pustego, $m_1$ 
    \item Wyznaczamy masę piknometru napełnionego cieczą wzorcową, $m_2$;
    \item Wyznaczamy masę piknometru napełnionego badaną cieczą, $m_3$;
\end{enumerate}
\subsection{Wyprowadzenie wzorów}
Wiedząc, że masę cieczy wzorcowej w piknometrze o objętości V możemy
wyliczyć jako:
\begin{equation}
    m_w = m_2 - m_1 = \rho_wV
\end{equation}
masa badanej cieczy wyniesie:
\begin{equation}
    m_c = m_3 - m_1 = \rho V
\end{equation}
Dzieląc równanie (8) przez równanie (9) otrzymujemy wzór na gęstość badanej cieczy:
\begin{equation}
    \rho = \frac{m_3 - m_1}{m_2 - m_1}\rho_w
\end{equation}
\subsection{Niepwność pomiarowa}
Niepewność pomiarową $w(f)$ wyznaczymy jako niepewność złożoną z pomiaru masy piknometru zawierającego ciecze różnego rodzaju. Przyjmujemy, że $\Delta r = 0.001$ (gdyż taką dokładność posiadała waga) laboratoryjna. Pochodne cząstkowe funkcji (10) względem $m_1$, $m_2$, $m_3$ wynoszą:
\begin{equation}
    \frac{\partial \rho}{\partial m_1} = \rho_w \cdot \frac{m_3 - m_2}{(m_2 - m_1)^2} = A
\end{equation}
\begin{equation}
    \frac{\partial \rho}{\partial m_2} = \rho_w \cdot \frac{-(m_3 - m_1)}{(m_2 - m_1)^2} = B   
\end{equation}
\begin{equation}
    \frac{\partial \rho}{\partial m_3} = \rho_w \cdot \frac{1}{m_2 - m_1} = C
\end{equation}
Ostatecznie niepwność pomiarowa dana jest wzorem:
{\large
    \begin{equation}
        w(f)=\sqrt{(A\Delta r)^2+(B\Delta r)^2+(C\Delta r)^2}
    \end{equation}
}
\subsection{Wyniki pomiarów}

Wyniki pomiarów mas wagą laboratoryjną przedstawiono w Tabeli 4:

\begin{table}[h!]
\centering
\begin{tabular}{|c|c|c|c|}
\hline
\textbf{Lp.} & \textbf{Woda destylowana} & \textbf{Denaturat} & \textbf{Woda słona}\\
\hline
1 & 79.435 & 69.718 & 85.489\\
2 & 79.468 & 69.742 & 85.508\\
3 & 79.456 & 69.739 & 85.514\\
4 & 79.409 & 69.740 & 85.519\\
5 & 79.443 & 69.733 & 85.498\\
\hline
\end{tabular}
\caption{Wyniki pomiarów cieczy w gramach}
\label{table:students}
\end{table}

Wyniki pomiaru piknometru przedstawiono w Tabeli 5:

\begin{table}[h!]
\centering
\begin{tabular}{|c|c|c|c|}
\hline
\textbf{Lp.} & \textbf{Piknometr}\\
\hline
1 & 28.509\\
2 & 28.468\\
3 & 28.501\\
4 & 28.509\\
5 & 28.504\\
\hline
\end{tabular}
\caption{Wyniki pomiaru piknometru w gramach}
\label{table:students}
\end{table}
\subsection{Wyniki doświadczenia}
Po zestawieniu danych pomiarowch z Tabel (4) i (5) razem z wzorami (10) i (14)
\subsection{Wnioski}

\end{document}
